\documentclass[a0, 36pt, plainboxedsections]{sciposter} % Available size are 14, 17, 20, 25, 30, 36...
% draft, final
\usepackage{zxjatype} \setjamainfont{YuGo-Medium} %YuGo-Medium
\usepackage{otf}
\usepackage{fontspec, fontawesome}
\usepackage[utf8]{inputenc}
\usepackage{geometry} \geometry{top=0.5cm, left=2cm, right=6cm, bottom=5cm}
\usepackage{amsmath, amssymb}
\usepackage{multicol}
\usepackage{tikz}
\usepackage[framemethod=tikz]{mdframed}
\usepackage{color, xcolor}
%\usepackage[american]{babel} %american, english
%\usepackage[ansinew]{inputenc}
%\usepackage[Mandalore]{sciposterpp}
\usepackage{setspace}
\usepackage{wasysym}
\usepackage[scaled=0.92]{helvet}
\usepackage{pifont}
\usepackage{colortbl}
\usepackage{float, wrapfig}
\usepackage[format=plain,labelformat=simple,labelsep=period,font=scriptsize]{caption}
\usepackage[T1]{fontenc}

%\renewcommand{\rmdefault}{qtm} % serif font
\renewcommand{\titlesize}{\Huge}
\renewcommand{\authorsize}{\large} % title, author, inst
\renewcommand{\baselinestretch}{1.2}
%%%%%%%%%%%%%%%%%%%%%%%%%%%%%%%%%%%%%%%%%%%%%%%%%%%%%%%%%%%%
% define color
\definecolor{mainCol}{RGB}{255, 255, 250} % BACKGROUND
\definecolor{TextCol}{HTML}{469B9A}
\definecolor{Black1}{HTML}{0D1412}
\definecolor{Blue1}{HTML}{469B9A} \definecolor{Blue2}{HTML}{84C0CC}
\definecolor{Orange1}{HTML}{F9B04F}
\definecolor{Grey1}{HTML}{D1D1D3}
%%%%%%%%%%%%%%%%%%%%%%%%%%%%%%%%%%%%%%%%%%%%%%%%%%%%%%%%%%%%
% define font awesome
\providecommand\faUniv{{\FA\symbol{"F19C}}}
% define mdframed style
\mdfdefinestyle{section.frame}{outerlinewidth=0, roundcorner=8pt, backgroundcolor=Blue1, linecolor=Blue1}
\mdfdefinestyle{subsection.frame}{outerlinewidth=6, roundcorner=8pt, backgroundcolor=Blue1!20, linecolor=Blue2}
\mdfdefinestyle{conclusion.frame}{outerlinewidth=8, roundcorner=8pt, backgroundcolor=Orange1!30,  linecolor=Orange1}
\mdfapptodefinestyle{subsection.frame}{rightline=true, innerleftmargin=16, innerrightmargin=16, 
frametitlerule=true, frametitlebackgroundcolor=Blue2}
\mdfapptodefinestyle{conclusion.frame}{rightline=true, innerleftmargin=16, innerrightmargin=16, 
frametitlerule=true, frametitlebackgroundcolor=Orange1}
% define multicol style
\setlength{\linewidth}{18pt}
\setlength{\columnsep}{2cm}
\def\columnseprulecolor{\color{Grey1}}
% define ruby
% ref) http://yasuda.homeip.net/insomnia/2011/07/sitatuki-ruby.html
\newcommand\uruby[3][0]{\leavevmode
  % 親文字とルビの寸法を取得
  \setbox0=\hbox{#2}\setbox1=\hbox{\tiny #3}%
  % 幅の大きいほうの寸法を \dimen0 に格納
  \ifdim\wd0>\wd1 \dimen0=\wd0\else\dimen0=\wd1\fi
  % \dimen1 に「ルビ高さ+深さ+間隔値」(下にずらす量)を設定
  \dimen1=\ht1 \advance\dimen1 \dp1 \advance\dimen1 #1\relax
  % \dimen0 の幅で親文字を出力し,ルビを \dimen1 寸法だけ下に下げる
  \hbox to\dimen0{\hfil#2\hfil}%
  \kern-\dimen0\raise-\dimen1\hbox{\vbox{\hbox to\dimen0{\tiny #3}}}}%
% define figure and table's caption style
\renewcommand{\figurename}{図}
\renewcommand{\tablename}{表}
\setlength{\fboxrule}{1pt}
%%%%%%%%%%%%%%%%%%%%%%%%%%%%%%%%%%%%%%%%%%%%%%%%%%%%%%%%%%%%
\rightlogo{images/logo.pdf}
\title{\textcolor{Blue1}{函南原生林を構成する樹種の株構造:\\\vspace{-0.2em} 葉群配置戦略としての多様な萌芽性}}
\author{\underline{瓜生真也}, 鄭 欣怡, 酒井暁子 (横浜国立大・院・環境情報) \normalsize{\faEnvelope \hspace{0.02em} \fontspec{HelveticaNeue-Italic}{suika1127@gmail.com}}}
%\institute{} \email{}
%\nologos
%%%%%%%%%%%%%%%%%%%%%%%%%%%%%%%%%%%%%%%%%%%%%%%%%%%%%%%%%%%%
\begin{document}
\conference{150319 日本生態学会第62回全国大会@鹿児島 [PA1-042]}
\maketitle
%%%%%%%%%%%%%%%%%%%%%%%%%%%%%%%%%%%%%%%%%%%%%%%%%%%%%%%%%%%%
\vspace{-2em}
\begin{multicols}{2}
\begin{mdframed}[style=section.frame]
  \flushleft\LARGE\textbf{\color{white}{はじめに}}
\end{mdframed}

\vspace{-0.6em}\subsection*{目的: 安定した成熟林を構成する樹種を対象に、\\\hspace*{6em}株構造と他の形質との関係を明らかにする}

株構造(幹のサイズ、本数)は種内間で異なる {\tiny(Escandón \textit{et al.} 2013; Bay \textit{et al.} 2014)}が、\textbf{\underline{群集内での株構造と共存機構のつながりは不明}}

\subsection*{仮説: 複数の幹をもつ種は...}

\vspace{-0.6em}\begin{figure}
 \begin{minipage}{0.4\hsize}
  \centering
   \includegraphics[scale=0.6]{images/tmp}  
 \end{minipage}
 \begin{minipage}{0.6\hsize}
  \ding{192}単幹で生育する種よりも\underline{\textbf{樹高が低い}}\\
  \ding{193}株内の\underline{\textbf{自己被陰を軽減する}}\\\hspace*{10.5em}\underline{\textbf{葉群配置を行う}}
 \end{minipage}
\end{figure}

\columnbreak
\begin{mdframed}[style=conclusion.frame,frametitle={\textbf{\Large{萌芽性は樹高と背反的に進化しており、\\\hspace*{1em}群集の中に多様な萌芽性の種が存在する}}}]
  \vspace{0.4em}
  \flushleft
  \normalsize{
  \textbf{\ding{192}多くの種が複数の幹をもつ; \ding{193}萌芽性と樹高はトレードオフ}
  
    \faCaretRight 高木種では萌芽幹をもたないか、わずかな大きさでしかない
    
    \faCaretRight 萌芽性がより強い種では株内の幹による自己被陰を軽減する
    
    \faCaretRight この傾向は対象種の\textbf{系統的偏りにより生じた偽相関ではない}
  }
  %\vspace{0.4em}
  \faHandLeft 群集内で、\\\textbf{\underline{より上層で光を獲得する}}と\textbf{\underline{下層でより効率的に光を受ける}} の戦略分化があり
  、\textbf{\underline{萌芽性は後者の戦術として進化}}している
 
\end{mdframed}

\end{multicols}
%%%%%%%%%%%%%%%%%%%%%%%%%%%%%%%%%%%%%%%%%%%%%%%%%%%%%%%%%%%%
\begin{mdframed}[style=section.frame]
  \flushleft\LARGE\textbf{\color{white}{方法}}
\end{mdframed}\vspace{-1.2em}

\begin{multicols}{2}

\begin{wrapfigure}{r}{12em}
  \centering
    \includegraphics[scale=0.27]{/Users/uri/Dropbox/LAB/Research/2013Structure_of_Multi-stemmed_Trees/Slide/talk_150206WS/images/kn25species.pdf} %3.0
\end{wrapfigure}

\subsection*{調査地: 静岡県函南原生林}

\faCaretRight 江戸時代より禁伐指定される成熟林

\faCaretRight 3標高地点にプロット(1.0ha)を設置

\faCaretRight 胸高直径(DBH) 5cm以上の毎木調査\footnote{プロットの設置、毎木調査は武生(東京農業大学), 澤田(東京農業大学, 現在は鹿児島大学), 磯谷(国士舘大学), 吉田(横浜国立大学)らの調査による}

\subsection*{調査対象: 14科17属25種}

\faCaretRight 1プロット内で20個体以上出現した種

\faCaretRight 個体数が最多となるプロットでのDBH上位60から90\%個体

\faCaretRight \textbf{\underline{系統独立対比}}: 系統に依存した形質の偽相関を対比値により検証\footnote{系統データはAPG I\hspace{-1pt}I\hspace{-1pt}Iのデータベース (R20120829.new \url{https://github.com/camwebb/tree-of-trees/blob/master/megatrees/R20120829.new}をもとにPhylomatic (Webb and Donoghue 2005 \url{http://phylodiversity.net/phylomatic/})にて作成}

\columnbreak
\subsection*{測定項目: 幹のサイズ、萌芽幹の被陰状態}

\vspace{-1em}
\begin{figure}
  \centering
	\includegraphics[scale=0.65]{images/stem_type_define.pdf}
\end{figure}

\vspace{-1.5em}
\subsection*{株構造の定量化: \uruby{株構造指数}{Stool Structure Index(SSI)} \scriptsize(Uryu \textit{et al.} unpublished)}

\vspace{-1.0em}
\begin{figure}
 \begin{minipage}{0.55\hsize}

個体㈱\footnote{\textbf{㈱は株式会社の略ではない}. \textbf{萌芽㈲}も有限会社の略ではなく、\textbf{萌芽幹を有する}を示す}内でBAが大きい順に並べる

\textbf{\underline{0.5 最小値}}: \textbf{すべての幹が同じ太さ}

\textbf{\underline{1.0 最大値}}: \textbf{主幹のみ}
 \end{minipage}
 \begin{minipage}{0.45\hsize}
  \vspace{-0.8em}\flushright
   \includegraphics[scale=1.0]{images/ex_stool_structure_index.pdf}  
 \end{minipage}
\end{figure}

\end{multicols}
%%%%%%%%%%%%%%%%%%%%%%%%%%%%%%%%%%%%%%%%%%%%%%%%%%%%%%%%%%%%
\begin{mdframed}[style=section.frame]
  \flushleft\LARGE\textbf{\color{white}{結果}}
\end{mdframed}\vspace{-1.2em}

\begin{multicols}{2}

\begin{mdframed}[style=subsection.frame,frametitle={\textbf{\huge{\ding{192}}\LARGE{株構造は種間で大きく異なる}}}]

\vspace{-1.0em}
\begin{figure}
  \centering
    \includegraphics[scale=1.5]{images/original/fig1_phylogeny_and_trait-1}
\end{figure}

\vspace{-1em}
\faCaretRight \textbf{複幹率}{\small(萌芽幹を持っている個体の割合): 平均}\textbf{43.43$\pm$}{\small 標準偏差}\textbf{36.02}

\hspace*{1.0em}\faHandLeft \textbf{\underline{0から100\%と幅広い多様性}}

\faCaretRight \textbf{株構造指数}: \textbf{0.89$\pm$0.12} {\small(萌芽㈲個体のみで計算)}

\hspace*{1.0em}\faHandLeft \textbf{\underline{多くの萌芽幹は主幹に対して相対的に小さい}}

\end{mdframed}

\begin{mdframed}[style=subsection.frame,frametitle=\textbf{\huge{\ding{194}}\LARGE{萌芽性が強いほど自己被陰を回避}}]

\vspace{-1.0em}
\begin{figure}
 \begin{minipage}{0.25\hsize}
  \centering
   \includegraphics[scale=0.45]{images/original/fig2_sprout_stem_shading_condition-1}
   
  %\vspace{-0.4em}
  \flushleft 
    {\tiny \textcolor[HTML]{4D4D4D}{\faStop} 被陰なし\\
    \textcolor[HTML]{C0C0C0}{\faStop} 他個体による被陰\\
    \vspace{-1.6em}\textcolor[HTML]{FFFFFF}{\faStop} 自個体内の他幹による被陰}
 \end{minipage}
 \begin{minipage}{0.75\hsize}
  \centering
   \includegraphics[scale=0.5]{images/original/fig3_sprout_shoot_self_shade-1}
   \includegraphics[scale=0.5]{images/original/fig3_sprout_shoot_self_shade_pic-1}
   
   \flushleft\vspace{-1em}\hspace*{0.6em}
    {\tiny \textcolor[HTML]{F9B04F}{\faCircle}: 落葉樹; \textcolor[HTML]{469B9A}{\faCircle}: 常緑樹.
		\hspace{0.6em}\textit{P} < 0.05: *; \textit{P} < 0.001: ***}
 \end{minipage}
\end{figure}

\end{mdframed}

\columnbreak
\begin{mdframed}[style=subsection.frame,frametitle=\textbf{\huge{\ding{193}}\LARGE{萌芽性が強くなるほど樹高が低下}}]

\vspace{-1.0em}
\begin{figure}
	\centering
		\includegraphics[scale=.94]{images/original/fig4_sprout_height-1}
	
	\vspace{-1.0em}\flushleft
		\hspace*{2.6em}{\footnotesize \textcolor[HTML]{F9B04F}{\faCircle}: 落葉樹; \textcolor[HTML]{469B9A}{\faCircle}: 常緑樹.
		\hspace{0.6em}Spearmanの順位相関 \textit{P} < 0.01: **; \textit{P} < 0.001: ***}
\end{figure}

\vspace{-1.0em}
\begin{figure}
 \begin{minipage}{0.5\hsize}
  \centering
   \includegraphics[scale=0.8]{images/original/fig4_sprout_height_pic-1}
   
   \vspace{-1.0em}{\tiny \textit{P} > 0.05: ns; \textit{P} < 0.05: *; \textit{P} < 0.01: **; \textit{P} < 0.001: ***}
 \end{minipage}
 \begin{minipage}{0.5\hsize}
{\large \faHandLeft \textbf{\underline{萌芽性と樹高の関係は}}\\ \textbf{\underline{系統的偏りによるもの}}\\\textbf{\underline{ではない}}}
 \end{minipage}
\end{figure}

\end{mdframed}

\end{multicols}

% https://github.com/brunobeltran/MatrixExp
% https://github.com/thmosqueiro/sciposterpp
% http://latexbr.blogspot.jp/2011/07/posters-cientificos-no-latex.html
% http://hikaru.6.ql.bz/memo.php?id=latex#sciposter
\end{document}