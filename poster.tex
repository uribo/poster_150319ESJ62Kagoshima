\documentclass[a0, 25, plainboxedsections]{sciposter} % Available size are 14, 17, 20, 25, 30, 36...
% draft, final
\usepackage{zxjatype} \setjamainfont{YuGo-Medium} % don't forget change to Helvetica.
\usepackage{otf}
\usepackage{fontspec, fontawesome}
\usepackage[utf8]{inputenc}
\usepackage{amsmath, amssymb}
\usepackage{multicol}
\usepackage{tikz}
\usepackage[framemethod=tikz]{mdframed}
\usepackage{color, xcolor}
\usepackage{setspace}
\usepackage{wasysym}
\usepackage[scaled=0.92]{helvet}
\usepackage{pifont}
\usepackage{colortbl}
\usepackage{float, wrapfig}
\usepackage[format=plain,labelformat=simple,labelsep=period,font=scriptsize]{caption}
\usepackage[T1]{fontenc}

%\renewcommand{\rmdefault}{qtm} % serif font
\renewcommand{\titlesize}{\Huge}
\renewcommand{\authorsize}{\Large} % title, author, inst
\renewcommand{\baselinestretch}{1.2}
%%%%%%%%%%%%%%%%%%%%%%%%%%%%%%%%%%%%%%%%%%%%%%%%%%%%%%%%%%%%
% define color
\definecolor{mainCol}{RGB}{255, 255, 250} % BACKGROUND
\definecolor{TextCol}{HTML}{469B9A}
\definecolor{Black1}{HTML}{0D1412}
\definecolor{Blue1}{HTML}{469B9A} \definecolor{Blue2}{HTML}{84C0CC}
\definecolor{Orange1}{HTML}{F9B04F}
\definecolor{Grey1}{HTML}{D1D1D3}
%%%%%%%%%%%%%%%%%%%%%%%%%%%%%%%%%%%%%%%%%%%%%%%%%%%%%%%%%%%%
% define font awesome
\providecommand\faUniv{{\FA\symbol{"F19C}}}
% define mdframed style
\mdfdefinestyle{section.frame}{outerlinewidth=0, roundcorner=8pt, backgroundcolor=Blue1, linecolor=Blue1}
\mdfdefinestyle{subsection.frame}{outerlinewidth=6, roundcorner=8pt, backgroundcolor=Blue1!20, linecolor=Blue2}
\mdfdefinestyle{conclusion.frame}{outerlinewidth=8, roundcorner=8pt, backgroundcolor=Orange1!30,  linecolor=Orange1}
\mdfapptodefinestyle{subsection.frame}{rightline=true, innerleftmargin=16, innerrightmargin=16, 
frametitlerule=true, frametitlebackgroundcolor=Blue2}
\mdfapptodefinestyle{conclusion.frame}{rightline=true, innerleftmargin=16, innerrightmargin=16, 
frametitlerule=true, frametitlebackgroundcolor=Orange1}
% define multicol style
\setlength{\linewidth}{18pt}
\setlength{\columnsep}{2cm}
\def\columnseprulecolor{\color{Grey1}}
\setlength{\fboxrule}{1pt}
%%%%%%%%%%%%%%%%%%%%%%%%%%%%%%%%%%%%%%%%%%%%%%%%%%%%%%%%%%%%
\rightlogo{images/logo.pdf}
\title{\textcolor{Blue1}{Alternative foliage strategies to relation stool structure among 25 old growth forest species}} %Stool structure variation of 25 temperate-forest tree species to a foliage distribution
\author{\underline{Shinya Uryu}, Xinyi Zheng, Akiko Sakai (Yokohama National University) \normalsize{\faEnvelope \hspace{0.02em} \fontspec{HelveticaNeue-Italic}{suika1127@gmail.com}}}
%%%%%%%%%%%%%%%%%%%%%%%%%%%%%%%%%%%%%%%%%%%%%%%%%%%%%%%%%%%%
\begin{document}
\conference{150319 The 62nd ESJ annual meeting@Kagoshima [PA1-042]}
\maketitle
%%%%%%%%%%%%%%%%%%%%%%%%%%%%%%%%%%%%%%%%%%%%%%%%%%%%%%%%%%%%
\vspace{-2em}
\begin{multicols}{2}
\begin{mdframed}[style=section.frame]
  \LARGE\textbf{\color{white}{\faRocket INTRODUCTION}}
\end{mdframed}

\vspace{-0.6em}\subsection*{Aim: Clearly to relationships between interspecific stool structure variation and architectural traits in a temperate old-growth forest.}

Intraspecific and interspecific stool structure patterns occur many species {\footnotesize(Escandón \textit{et al.} 2013; Bay \textit{et al.} 2014)}, but it is \textbf{\underline{unclear how these effects community levels.}}

\vspace{-0.6em}\begin{figure}
 \begin{minipage}{0.6\hsize}
 \subsection*{Hypotheses: Species with high ability to sprouting are...}
  \ding{192} \underline{\textbf{tended to be short}} because cost of height growth, and
  
  \ding{193} \underline{\textbf{exposed to the problem}} of self-shading from other stem.

 \end{minipage}
 \begin{minipage}{0.4\hsize}
  \centering
   \includegraphics[scale=0.5]{images/sprout_merit_demerit_e}
 \end{minipage}
\end{figure}



\columnbreak
\begin{mdframed}[style=conclusion.frame,frametitle={\textbf{\Large{\faFlagAlt \vspace{0.02em} SYNTHESIS: {Sprout ability is evolved to against tree height, and promoted species coexistence.}}}}]
  \vspace{0.4em}
  \flushleft
  \normalsize{
  \textbf{Almost of species are multi-stem; Evidence of a sprouting vs. height trade-off.}
  
    \faCaretRight Canopy tree otherwise deciduous sprout stem was relatively small.
    
    \faCaretRight More highly sprout ability species was more avoid self-shading from other stems.
    
    \faCaretRight This pattern was also confirmed by the phylogenetic regression analyses based on phylogenetic contrasts.
  }
  
  \vspace{0.4em}
  \faHandLeft These indicating specialization for light that differential preferences of species for either capopy or forest understorey \textbf{\underline{sprout ability was developed that as the latter tactics.}}
  
 
 
\end{mdframed}

\end{multicols}
%%%%%%%%%%%%%%%%%%%%%%%%%%%%%%%%%%%%%%%%%%%%%%%%%%%%%%%%%%%%
\begin{mdframed}[style=section.frame]
  \LARGE\textbf{\color{white}{\faBeaker MATERIALS AND METHODS}}
\end{mdframed}\vspace{-1.2em}

\begin{multicols}{2}

\begin{wrapfigure}{r}{16.7em}
  \centering
    \includegraphics[scale=0.2]{/Users/uri/Dropbox/LAB/Research/2013Structure_of_Multi-stemmed_Trees/Slide/talk_150206WS/images/kn25species.pdf} %3.0
\end{wrapfigure}

\subsection*{Site: Kannami old growth forest}

\faCaretRight For the last 400 years, the forest has been protected\\ from anthropogenic alterations.

\faCaretRight Three permanent 1.0-ha plots, 100 m apart from 600 to 800m.

\faCaretRight All tree individuals $\geq$ 5 cm dbh were identified, measured, and marked.%\footnote{プロットの設置、毎木調査は東京農業大学 武生、澤田(現在、鹿児島大学)、国士舘大学 磯谷、横浜国立大学 吉田らの調査による}

\subsection*{Species selection: 25 species from 17 genera 14 families}

\faCaretRight Commonly found in the forest (20 individuals per plot).

\faCaretRight Mature trees: DBH between 60 to 90\%

\faCaretRight \textbf{\underline{Phylogenetically independent contrasts: PICs}} %系統関係に依存した形質値の偽相関を対比値を用いて検証\footnote{系統データはAPG I\hspace{-1pt}I\hspace{-1pt}Iのデータベース (R20120829.new \url{https://github.com/camwebb/tree-of-trees/blob/master/megatrees/R20120829.new}をもとにPhylomatic (Webb and Donoghue 2005 \url{http://phylodiversity.net/phylomatic/})にて作成}

\columnbreak
\subsection*{Investment: Stem size and shading}

\begin{figure}
  \centering
	\includegraphics[scale=0.6]{images/stem_type_define_e.pdf}
\end{figure}

\vspace{-1.8em}\begin{figure}
 \begin{minipage}{0.6\hsize}
 \subsection*{Stool Structure Index: SSI \tiny(Uryu \textit{et al.} unpublished)}

\textbf{\underline{0.5 min.}}: \textbf{each stem are equality size}

\textbf{\underline{1.0 max.}}: \textbf{single-stem}
 \end{minipage}
 \begin{minipage}{0.4\hsize}
  \centering
   \includegraphics[scale=0.8]{images/ex_stool_structure_index_e}  
 \end{minipage}
\end{figure}

\end{multicols}
%%%%%%%%%%%%%%%%%%%%%%%%%%%%%%%%%%%%%%%%%%%%%%%%%%%%%%%%%%%%
\begin{mdframed}[style=section.frame]
  \LARGE\textbf{\color{white}{\faLightbulb RESULTS}}
\end{mdframed}\vspace{-1.2em}

\begin{multicols}{2}

\begin{mdframed}[style=subsection.frame,frametitle={\textbf{\LARGE{Strong variation in stool structure among species}}}]

%\vspace{-1.0em}
\begin{figure}
  \centering
    \includegraphics[scale=1.5]{images/original/fig1_phylogeny_and_trait-1e}
\end{figure}

\vspace{-1em}
\faCaretRight \textbf{Multi-stem ratio} Mean43.43$\pm$SD36.02

\hspace*{2.5em}{\large \faHandLeft \textbf{\underline{broad range from 0 to 100\%}}}

{\small
\hspace*{2.5em}Non-sprout: \textit{Fagus crenata, Zelkova serrata, Acer amoenum, Stewartia monadelpha, Carpinus tschonoskii} (Deciduos tree)
}

\faCaretRight \textbf{SSI}: Mean0.89$\pm$SD0.12}

\hspace*{2.5em}{\large \faHandLeft \textbf{\underline{Almost sprout stem was relatively small than main stem.}}}

\end{mdframed}

\begin{mdframed}[style=subsection.frame,frametitle=\textbf{\LARGE{More highly sprouters are more avoid self-shading}}]

\vspace{-0.5em}
\begin{figure}
 \begin{minipage}{0.25\hsize}
  \centering
   \includegraphics[scale=0.5]{images/original/fig2_sprout_stem_shading_condition-1e}
   
  \vspace{-0.4em}\flushleft 
    {\tiny \textcolor[HTML]{4D4D4D}{\faStop} No shading\\
    \textcolor[HTML]{C0C0C0}{\faStop} Other shading\\
    \vspace{-1.6em}\textcolor[HTML]{FFFFFF}{\faStop} Self shading}
 \end{minipage}
 \begin{minipage}{0.75\hsize}
  \centering
   \includegraphics[scale=.5]{images/original/fig3_sprout_shoot_self_shade-1e}
   \includegraphics[scale=.5]{images/original/fig3_sprout_shoot_self_shade_pic-1e}
   
   \flushleft\vspace{-1em}\hspace*{0.6em}
    {\tiny \textit{P} < 0.05: *; \textit{P} < 0.001: ***}
 \end{minipage}
\end{figure}

\end{mdframed}

\columnbreak
\begin{mdframed}[style=subsection.frame,frametitle=\textbf{\LARGE{Sprout avility increases with decreasing height}}] %Potential sprout was significantly negatively correlated with height
\vspace{-1.0em}
\begin{figure}
	\centering
		\includegraphics[scale=0.9]{images/original/fig4_sprout_height-1e}
	
	\flushleft
		\hspace*{2.6em}{\footnotesize \textcolor[HTML]{F9B04F}{\faCircle}: Deciduous; \textcolor[HTML]{469B9A}{\faCircle}: Evergreen.
		\hspace{0.6em}Spearman's rank correlation \textit{P} < 0.01: **; \textit{P} < 0.001: ***}
\end{figure}

\vspace{-1.5em}
\begin{figure}
 \begin{minipage}{0.55\hsize}
  \centering
   \includegraphics[scale=0.8]{images/original/fig4_sprout_height_pic-1e}
   
   {\tiny \textit{P} > 0.05: ns; \textit{P} < 0.05: *; \textit{P} < 0.01: **; \textit{P} < 0.001: ***}
 \end{minipage}
 \begin{minipage}{0.45\hsize}
 
 
{\large \faHandLeft \textbf{\underline{Correlations between}}\\ \textbf{\underline{sprout ability and height}}\\\textbf{\underline{based on PICs paralleled}}\\\textbf{\underline{the results across species.}}}
 \end{minipage}
\end{figure}

\end{mdframed}

\end{multicols}

% https://github.com/brunobeltran/MatrixExp
% https://github.com/thmosqueiro/sciposterpp
% http://latexbr.blogspot.jp/2011/07/posters-cientificos-no-latex.html
% http://hikaru.6.ql.bz/memo.php?id=latex#sciposter
\end{document}