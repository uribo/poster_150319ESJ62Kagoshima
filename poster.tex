\documentclass[a0, 30pt, plainboxedsections]{sciposter} % Available size are 14, 17, 20, 25, 30, 36...
\usepackage{zxjatype} \setjamainfont{YuGo-Medium} %YuGo-Medium
\usepackage{fontspec, fontawesome}
\usepackage[utf8]{inputenc}
%\usepackage{geometry} \geometry{top=1cm, left=2cm, right=6cm}
\usepackage{amsmath, amssymb}
\usepackage{multicol}
\usepackage{tikz}
\usepackage[framemethod=tikz]{mdframed}
\usepackage{color, xcolor}
%\usepackage[american]{babel} %american, english
%\usepackage[ansinew]{inputenc}
%\usepackage[Mandalore]{sciposterpp}
\usepackage{setspace}
\usepackage{wasysym}
\usepackage[scaled=0.92]{helvet}
\usepackage{pifont}
\usepackage{colortbl}
\usepackage{float, wrapfig}
\usepackage[format=plain,labelformat=simple,labelsep=period,font=scriptsize]{caption}
\usepackage[T1]{fontenc}

%\renewcommand{\rmdefault}{qtm} % serif font
%\renewcommand{\titlesize}{\Huge}
\renewcommand{\authorsize}{\Large} % title, author, inst
\renewcommand{\baselinestretch}{1.2}
%%%%%%%%%%%%%%%%%%%%%%%%%%%%%%%%%%%%%%%%%%%%%%%%%%%%%%%%%%%%
% define color
\definecolor{mainCol}{RGB}{255, 255, 250} % BACKGROUND
\definecolor{TextCol}{HTML}{469B9A}
\definecolor{Black1}{HTML}{0D1412}
\definecolor{Blue1}{HTML}{469B9A} \definecolor{Blue2}{HTML}{84C0CC}
\definecolor{Orange1}{HTML}{F9B04F}
\definecolor{Grey1}{HTML}{D1D1D3}
%%%%%%%%%%%%%%%%%%%%%%%%%%%%%%%%%%%%%%%%%%%%%%%%%%%%%%%%%%%%
% define font awesome
\providecommand\faUniv{{\FA\symbol{"F19C}}}
% define mdframed style
\mdfdefinestyle{section.frame}{outerlinewidth=0, roundcorner=8pt, backgroundcolor=Blue1, linecolor=Blue1}
\mdfdefinestyle{subsection.frame}{outerlinewidth=6, roundcorner=8pt, backgroundcolor=Blue1!20, linecolor=Blue2}
\mdfdefinestyle{conclusion.frame}{outerlinewidth=8, roundcorner=8pt, backgroundcolor=Orange1!30,  linecolor=Orange1}
\mdfapptodefinestyle{subsection.frame}{rightline=true, innerleftmargin=16, innerrightmargin=16, 
frametitlerule=true, frametitlebackgroundcolor=Blue2}
\mdfapptodefinestyle{conclusion.frame}{rightline=true, innerleftmargin=16, innerrightmargin=16, 
frametitlerule=true, frametitlebackgroundcolor=Orange1}
% define multicol style
\setlength{\linewidth}{18pt}
\setlength{\columnsep}{2cm}
\def\columnseprulecolor{\color{Grey1}}
%%%%%%%%%%%%%%%%%%%%%%%%%%%%%%%%%%%%%%%%%%%%%%%%%%%%%%%%%%%%
\rightlogo{images/neac.pdf} %/Users/uri/Dropbox/git/root-sucker/Resources/root-sucker-icon.png
\title{\textcolor{Blue1}{函南原生林を構成する樹種の株構造:\\\vspace{-0.2em} 葉群配置戦略としての多様な萌芽性}}
\author{\underline{瓜生真也}, 鄭 欣怡, 酒井暁子 (横浜国立大・院・環境情報) \normalsize{\faEnvelope \hspace{0.02em} \fontspec{GillSans-Italic}{suika1127@gmail.com}}}
%\institute{} \email{}
%\nologos
%%%%%%%%%%%%%%%%%%%%%%%%%%%%%%%%%%%%%%%%%%%%%%%%%%%%%%%%%%%%
\begin{document}
\conference{150319 日本生態学会第62回全国大会@鹿児島 [PA1-042]}
\maketitle
%%%%%%%%%%%%%%%%%%%%%%%%%%%%%%%%%%%%%%%%%%%%%%%%%%%%%%%%%%%%
\vspace{-2em}
\begin{multicols}{2}
\begin{mdframed}[style=section.frame]
  \centering\huge\textbf{\color{white}{はじめに}}
\end{mdframed}

\subsection*{目的: 森林樹木群集を対象に、株構造の差異に働く\\他の性質との関係を明らかにする}

\begin{figure}
  \centering\includegraphics[scale=1.0]{images/sprout_merit_demerit.pdf}
\end{figure}

\vspace{-0.8em}
\subsection*{仮説: 複数の幹をもつ種は...}

\begin{enumerate}\setlength{\itemindent}{1em}
\item 単幹で生育する種よりも\underline{\textbf{樹高が低く}}なる
\item 多幹化に伴う\underline{\textbf{自己被陰の影響を軽減できる葉群配置}}を行う
\end{enumerate}

\columnbreak
\begin{mdframed}[style=conclusion.frame,frametitle={\huge\textbf{\color{Black1}{
  \large\textbf{\faFlagAlt \vspace{0.02em} 結論: {萌芽性は最大樹高と相反して発達し、多様な受光体制の存在が樹種の共存に貢献している}}}}}]
  \vspace{0.4em}
  \flushleft
  \normalsize{\underline{結果\ding{192}}森林樹木の多くの種は複数の幹をもつ\\
  \underline{結果\ding{193}}樹高と萌芽特性はトレードオフの関係にある\\
  }
  \large{\faHandLeft \vspace{0.02em} 多幹化と低木化は\textbf{\underline{系統関係に依存するものではない}}ことを意味する} % 連動して
  
  \vspace{0.4em}
  \normalsize{\underline{結果\ding{193}}高木種(とくに落葉樹)は萌芽幹をもたない・主幹に対してわずかな大きさしかない\\
  \underline{結果\ding{194}}より萌芽性が強い種では株内の幹による自己被陰を軽減する傾向\\
  }
  \large{\faHandLeft \vspace{0.02em} 樹木の多様な萌芽性として「\textbf{\underline{より上層で光を獲得する}}」と「\textbf{\underline{下層でより効率的に光を受ける}}」という戦略分化が森林内でみられる
  } % 多幹性は進化
\end{mdframed}
\end{multicols}
%%%%%%%%%%%%%%%%%%%%%%%%%%%%%%%%%%%%%%%%%%%%%%%%%%%%%%%%%%%%
\begin{mdframed}[style=section.frame]
  \centering\huge\textbf{\color{white}{方法}}
\end{mdframed}

\begin{multicols}{3}\footnotesize{

\subsection*{\small{調査地: 静岡県函南原生林}}

\begin{itemize}\setlength{\itemindent}{1em}
  \item 禁伐指定により人為的撹乱の影響が少ない成熟林
  \item 3つの標高地点にプロット(1.0ha)を設置
  \begin{itemize}\setlength{\itemindent}{1em}
    \item 常緑樹が優占する600m, 落葉樹優占の800m, \newline 常緑・落葉移行帯の700m
    \item 胸高直径(DBH) 5cm以上を対象に毎木調査\footnote{プロットの設置、毎木調査は東京農業大学 武生、現鹿児島大学 澤田らの調査による}
  \end{itemize}
\end{itemize}

\subsection*{\small{対象種および対象個体}}

\begin{itemize}\setlength{\itemindent}{1em}
  \item 各種の個体数が最も多い標高地点でサンプリング
  \item プロット内で個体数の多い25種
  \begin{itemize}\setlength{\itemindent}{1em}
    \item 落葉樹15種、常緑樹10種
    \item 高木11種、亜高木12種、低木2種
  \end{itemize}
  \item 各種のDBH上位60 - 90\%の成熟個体
\end{itemize}

\columnbreak
\subsection*{\small{測定項目}}

\vspace{-1em}
\begin{figure}
	\centering
	\includegraphics[scale=0.8]{images/stem_type_define.pdf}
\end{figure}

\subsection*{\small{株構造指数}}

\begin{figure}
  \flushleft
   \includegraphics{images/stool_structure_index2.pdf}
\end{figure}

\columnbreak
\subsection*{\small{系統解析: 系統独立対比}}

%\vspace{-3em}
\begin{wrapfigure}{r}{16em}
  \includegraphics[scale=0.8]{/Users/uri/Dropbox/git/Data-analyses/life-history_traits_and_stool_structure/Figure/kn25_phylogenetic_tree_phylogram-1.pdf}
\end{wrapfigure}

Phylomatic(Webb and Donoghue 2005)を用いてAPG III系統樹を作成、

解析時、形質間の系統的な偏りを補正するため、系統独立対比を実施

これらの解析はすべてR 3.1.2で実行。

}\end{multicols}
%%%%%%%%%%%%%%%%%%%%%%%%%%%%%%%%%%%%%%%%%%%%%%%%%%%%%%%%%%%%
\begin{mdframed}[style=section.frame]
  \centering\huge\textbf{\color{white}{結果}}
\end{mdframed}

\begin{multicols}{2}

\renewcommand{\baselinestretch}{0.8}
\begin{mdframed}[style=subsection.frame,frametitle={\huge\textbf{\color{Black1}{\LARGE{\ding{192}}\Large{25種の株構造は種間で大きく異なる}}}}]
\renewcommand{\baselinestretch}{1.2}

\subsection*{各種の株構造と全種の萌芽特性の平均および標準偏差}

\begin{wraptable}{r}{18em}\caption{調査対象25種の要約。複幹率が高い種順で並べ、上位および下位の5種の行を塗りつぶした。}\vspace{-0.4em}
  \scalebox{0.38}{\begin{tabular}{llrrrrrlll}
  \hline
\rowcolor{white}種名 & 略称 & 個体数 & 複幹率 & 萌芽本数 & 株構造指数 & 幹BA比 & 葉特性 & 生活型 & 標高 \\ 
  \hline
\rowcolor{Orange1!30}チドリノキ & Ac &  22 & 100.00 & 8.27 & 0.86 & 0.43 & 落葉 & 亜高木 & 700m \\ 
\rowcolor{Orange1!30}アブラチャン & Lp &  31 & 100.00 & 5.87 & 0.69 & 0.62 & 落葉 & 亜高木 & 800m \\ 
\rowcolor{Orange1!30}ヒイラギ & Oh &  22 & 81.80 & 2.50 & 0.91 & 0.09 & 常緑 & 亜高木 & 600m \\ 
\rowcolor{Orange1!30}イヌガシ & Na &  27 & 81.50 & 4.52 & 0.96 & 0.18 & 常緑 & 高木 & 600m \\ 
\rowcolor{Orange1!30}カマツカ & Pv &  20 & 80.00 & 1.55 & 0.49 & 0.21 & 落葉 & 低木 & 800m \\ 
\rowcolor{white}ヤブニッケイ & Cy &  27 & 77.80 & 2.15 & 0.88 & 0.12 & 常緑 & 高木 & 600m \\ 
\rowcolor{white}アオキ & Aj &  21 & 76.20 & 2.81 & 0.85 & 0.18 & 常緑 & 低木 & 600m \\ 
\rowcolor{white}アオハダ & Im &  23 & 73.90 & 3.35 & 0.87 & 0.16 & 落葉 & 亜高木 & 800m \\ 
\rowcolor{white}ヒサカキ & Ej &  22 & 68.20 & 3.09 & 0.93 & 0.07 & 常緑 & 亜高木 & 600m \\ 
\rowcolor{white}ウラジロガシ & Qs &  21 & 66.70 & 1.81 & 0.95 & 0.02 & 常緑 & 高木 & 600m \\ 
\rowcolor{white}シキミ & Ia &  30 & 63.30 & 3.40 & 0.96 & 0.03 & 常緑 & 亜高木 & 800m \\ 
\rowcolor{white}イヌツゲ & Ic &  26 & 53.80 & 1.38 & 0.95 & 0.01 & 常緑 & 亜高木 & 800m \\ 
\rowcolor{white}マメザクラ & Ci &  23 & 47.80 & 1.57 & 0.86 & 0.10 & 落葉 & 亜高木 & 700m \\ 
\rowcolor{white}アカガシ & Qa &  27 & 37.00 & 1.44 & 0.89 & 0.09 & 常緑 & 高木 & 600m \\ 
\rowcolor{white}カジカエデ & Ad &  25 & 28.00 & 0.84 & 0.90 & 0.03 & 落葉 & 亜高木 & 800m \\ 
\rowcolor{white}シロダモ & Ns &  25 & 24.00 & 0.60 & 0.98 & 0.00 & 常緑 & 亜高木 & 600m \\ 
\rowcolor{white}タンナサワフタギ & Sc &  23 & 13.00 & 0.13 & 0.99 & 0.01 & 落葉 & 亜高木 & 800m \\ 
\rowcolor{white}イロハモミジ & Ap &  21 & 4.80 & 0.05 & 0.94 & 0.02 & 落葉 & 高木 & 600m \\ 
\rowcolor{white}イタヤカエデ & Am &  24 & 4.20 & 0.04 & 1.00 & 0.01 & 落葉 & 高木 & 700m \\ 
\rowcolor{white}シラキ & Nj &  27 & 3.70 & 0.04 & 1.00 & 0.02 & 落葉 & 亜高木 & 800m \\ 
\rowcolor{Blue2!80}ブナ & Fc &  27 & 0.00 & 0.00 & 1.00 & 0.00 & 落葉 & 高木 & 700m \\ 
\rowcolor{Blue2!80}ケヤキ & Zs &  27 & 0.00 & 0.00 & 1.00 & 0.00 & 落葉 & 高木 & 700m \\ 
\rowcolor{Blue2!80}オオモミジ & Aa &  26 & 0.00 & 0.00 & 1.00 & 0.00 & 落葉 & 高木 & 700m \\ 
\rowcolor{Blue2!80}ヒメシャラ & Sm &  21 & 0.00 & 0.00 & 1.00 & 0.00 & 落葉 & 高木 & 800m \\ 
\rowcolor{Blue2!80}イヌシデ & Ct &  28 & 0.00 & 0.00 & 1.00 & 0.00 & 落葉 & 高木 & 800m \\ 
   \hline
\end{tabular}}
\end{wraptable}

\textbf{\underline{複幹率}}\\{\footnotesize(複幹を持っている個体の割合)}\\... 43.43$\pm$36.02\\
{\scriptsize 亜高木のチドリノキ、アブラチャンでは100\%、ブナやケヤキなど、すべての高木落葉樹種は0\%}\\
\textbf{\underline{萌芽本数}}{\footnotesize(複幹率0\%より高い)}\\... 2.27$\pm$2.11\\
{\scriptsize 最大はチドリノキの8.27本}\\
\textbf{\underline{株構造指数}}{\footnotesize(複幹率0\%より高い)}\\... 0.89$\pm$0.12\\
\textbf{\underline{萌芽幹BA比}}{\footnotesize(複幹率0\%より高い)}\\... 0.12$\pm$0.16

\end{mdframed}

\renewcommand{\baselinestretch}{0.8}
\begin{mdframed}[style=subsection.frame,frametitle=
  \huge\textbf{\color{Black1}{\LARGE{\ding{194}}\Large{萌芽性が強い種ほど自己被陰率が下がる}}}]
\renewcommand{\baselinestretch}{1.2}

\subsection*{複幹率5\%以上の17種を対象とした幹の被陰状況と萌芽性}

多くの萌芽幹は主幹を中心とした個体内の他の幹によって被陰されていた。一方で萌芽性が高い種ほど自己被陰の割合が減少する傾向が系統に関わらずみられた。

\begin{figure}
 \begin{minipage}{0.48\hsize}
  \centering
   \includegraphics[scale=1.5]{images/fig3self_shade_ratio_barplot.pdf}
 \end{minipage}
 \begin{minipage}{0.25\hsize}
  \centering
   \includegraphics[scale=0.8]{images/fig4self_shad_ssi_cor.pdf}
 \end{minipage}
 \begin{minipage}{0.25\hsize}
  \centering
   \includegraphics[scale=0.8]{images/fig5self_shad_ssi_cor_pic.pdf}
 \end{minipage}
\end{figure}

\end{mdframed}

\columnbreak
\renewcommand{\baselinestretch}{0.8}
\begin{mdframed}[style=subsection.frame,frametitle=
% 樹高と萌芽特性
  \huge\textbf{\color{Black1}{\LARGE{\ding{193}}\Large{樹高が高くなるほど萌芽性が強くなる}}}]
\renewcommand{\baselinestretch}{1.2} %revert baseline

\subsection*{樹高と萌芽特性との関係}

Spearmanの順位相関係数

\begin{figure}
	\centering
		\includegraphics[scale=1.8]{images/fig1height_spr_cor.pdf}
\end{figure}

系統独立対比による解析でも同様の傾向を検出

\begin{figure}
	\centering
	  \includegraphics[scale=1.8]{images/fig2height_spr_cor_pic.pdf}
\end{figure}
\end{mdframed}

\end{multicols}

% https://github.com/brunobeltran/MatrixExp
% https://github.com/thmosqueiro/sciposterpp
% http://latexbr.blogspot.jp/2011/07/posters-cientificos-no-latex.html
% http://hikaru.6.ql.bz/memo.php?id=latex#sciposter
\end{document}