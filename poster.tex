\documentclass[a0, 36pt, plainboxedsections]{sciposter} % Available size are 14, 17, 20, 25, 30, 36...
\usepackage{zxjatype} \setjamainfont{YuGo-Medium} %YuGo-Medium
\usepackage{fontspec, fontawesome}
\usepackage[utf8]{inputenc}
%\usepackage{geometry} \geometry{top=1cm, left=2cm, right=6cm}
\usepackage{amsmath, amssymb}
\usepackage{multicol}
\usepackage{tikz}
\usepackage[framemethod=tikz]{mdframed}
\usepackage{color, xcolor}
%\usepackage[american]{babel} %american, english
%\usepackage[ansinew]{inputenc}
%\usepackage[Mandalore]{sciposterpp}
\usepackage{setspace}
\usepackage{wasysym}
\usepackage[scaled=0.92]{helvet}
\usepackage{pifont}
\usepackage{colortbl}
\usepackage{float, wrapfig}
\usepackage[T1]{fontenc}

%\renewcommand{\rmdefault}{qtm} % serif font
\renewcommand{\authorsize}{\Large} % title, author, inst
\renewcommand{\baselinestretch}{1.2}
%%%%%%%%%%%%%%%%%%%%%%%%%%%%%%%%%%%%%%%%%%%%%%%%%%%%%%%%%%%%
% define color
\definecolor{mainCol}{RGB}{255, 255, 250} % BACKGROUND
\definecolor{TextCol}{HTML}{469B9A}
\definecolor{Black1}{HTML}{0D1412}
\definecolor{Blue1}{HTML}{469B9A} \definecolor{Blue2}{HTML}{84C0CC}
\definecolor{Orange1}{HTML}{F9B04F}
\definecolor{Grey1}{HTML}{D1D1D3}
%%%%%%%%%%%%%%%%%%%%%%%%%%%%%%%%%%%%%%%%%%%%%%%%%%%%%%%%%%%%
% define font awesome
\providecommand\faUniv{{\FA\symbol{"F19C}}}
% define mdframed style
\mdfdefinestyle{section.frame}{outerlinewidth=0, roundcorner=8pt, backgroundcolor=Blue1, linecolor=Blue1}
\mdfdefinestyle{subsection.frame}{outerlinewidth=0, roundcorner=8pt, backgroundcolor=Blue2, linecolor=Blue2}
\mdfdefinestyle{conclusion.frame}{outerlinewidth=8, roundcorner=5pt, align=left, backgroundcolor=Orange1!30,  linecolor=Orange1}
%%%%%%%%%%%%%%%%%%%%%%%%%%%%%%%%%%%%%%%%%%%%%%%%%%%%%%%%%%%%
\title{\textcolor{Blue1}{函南原生林を構成する樹種の株構造:\\葉群配置戦略としての多様な萌芽性}}
\author{\faGroup \hspace{0.02em} \underline{瓜生真也}, 鄭 欣怡, 酒井暁子 (\faUniv \hspace{0.02em} 横浜国立大・院・環境情報) \normalsize{\faEnvelope \hspace{0.02em} \fontspec{ComicSansMS}{suika1127@gmail.com}}}
%\institute{} \email{}
\nologos
%%%%%%%%%%%%%%%%%%%%%%%%%%%%%%%%%%%%%%%%%%%%%%%%%%%%%%%%%%%%
\begin{document}
\conference{150319 日本生態学会第62回全国大会@鹿児島 [PA1-042]}
\maketitle
%%%%%%%%%%%%%%%%%%%%%%%%%%%%%%%%%%%%%%%%%%%%%%%%%%%%%%%%%%%%
\vspace{-2em}
\begin{multicols}{2}
\begin{mdframed}[style=section.frame]
  \centering\huge\textbf{\color{white}{\faRocket \vspace{0.02em} はじめに}}
\end{mdframed}

\subsection*{目的: 森林樹木群集を対象に、多幹性の差異に働く要因を明らかにする} % 種の生活型や個体の生育環境に応じて多様な反応を示す多幹性とその決定要因を明らかにする %多幹性を左右すると考えられる仮説を検証し、 %樹木の多幹性が群集に及ぼす効果を評価する %樹木群集における多幹性の進化に働く要因を評価する

\begin{figure}
  \center\includegraphics[scale=1.0]{images/image.pdf}  
\end{figure}
% 林冠木の多幹化は樹高成長の妨げ、林冠下では複数の萌芽幹からなる株構造が受光効率を促進という2面性
%複数の幹からなる株構造では、加えて個体内での自己被陰が問題

\vspace{-0.8em}
\subsection*{仮説: 複数の幹をもつ種は...}

\begin{enumerate}\setlength{\itemindent}{1em}
\item 単幹で生育する種よりも\underline{\textbf{樹高が低く}}なる
\item 多幹化に伴う\underline{\textbf{自己被陰の影響を軽減できる葉群配置}}を行う
\end{enumerate}

\columnbreak
\begin{mdframed}[style=conclusion.frame]
  \large\textbf{\faFlagAlt \vspace{0.02em} 結論: {多幹性は最大樹高と相反して発達し、多様な受光体制の存在が樹種の共存に貢献している}} %多幹化に伴う
  \vspace{0.4em}
  \flushleft
  \normalsize{\underline{結果\ding{192}}森林樹木の萌芽性は種によりさまざまであった\\
  \underline{結果\ding{193}}系統に依存せず、萌芽幹への投資と樹高は負の関係\\
  }
  \large{\faHandLeft \vspace{0.02em} 多幹化と高木化は\underline{\textbf{背反的に進化している}}ことを示唆する}
  
  \vspace{0.4em}
  \normalsize{\underline{結果\ding{194}}多くの萌芽幹は自己被陰される。一方、萌芽に多くの資源を配分する種では自己被陰を回避する傾向\\
  }
  \large{\faHandLeft \vspace{0.02em} 多様な萌芽性は「\underline{\textbf{より上層で光を獲得する性質}}」と「\underline{\textbf{下層でより効率的に光を受ける戦術}}」として葉群配置に影響する
  } % 多幹性は進化
\end{mdframed}
\end{multicols}
%%%%%%%%%%%%%%%%%%%%%%%%%%%%%%%%%%%%%%%%%%%%%%%%%%%%%%%%%%%%
\begin{mdframed}[style=section.frame]
  \centering\huge\textbf{\color{white}{\faLightbulb \vspace{0.02em} 結果}}
\end{mdframed}

\begin{multicols}{2}

\renewcommand{\baselinestretch}{0.8}
\begin{mdframed}[style=subsection.frame]
% 25種の株構造
  \huge\textbf{\color{Black1}{\Large{\ding{192}25種の株構造は種間で大きく異なる}}}
\end{mdframed}
\renewcommand{\baselinestretch}{1.2}

\begin{wrapfigure}{r}{14em}
  \scalebox{0.30}{\begin{tabular}{llrrrrlrlrl} \hline
\rowcolor{Grey1!40}種名 & 略称 & 個体数 & 複幹率 & 萌芽本数 & 幹BA比 & 葉特性 & 最大樹高 & 生活型 & AUC & 主要な標高 \\ \hline
\rowcolor{Orange1!30}チドリノキ  &  Ac &  22 & \textbf{\textcolor{Orange1}{100.00}} & \textbf{\textcolor{Orange1}{8.27}} & 0.57 & 落葉樹 & 10.02 & 亜高木 & 0.86 & 700m \\ 
\rowcolor{Orange1!30}アブラチャン  &  Lp &  31 & \textbf{\textcolor{Orange1}{100.00}} & 5.87 & \textbf{\textcolor{Blue2}{0.38}} & 落葉樹 & 9.26 & 亜高木 & 0.69 & 800m \\ 
\rowcolor{Orange1!30}ヒイラギ  &  Oh &  22 & 81.80 & 2.50 & 0.91 & 常緑樹 & 8.27 & 亜高木 & 0.91 & 600m \\ 
\rowcolor{Orange1!30}イヌガシ  &  Na &  27 & 81.50 & 4.52 & 0.82 & 常緑樹 & 14.23 & 高木 & 0.96 & 600m \\ 
\rowcolor{Orange1!30}カマツカ  &  Pv &  20 & 80.00 & 1.55 & 0.79 & 落葉樹 & 6.06 & 低木 & \textbf{\textcolor{Blue2}{0.49}} & 800m \\ 
ヤブニッケイ  &  Cy &  27 & 77.80 & 2.15 & 0.88 & 常緑樹 & 13.42 & 高木 & 0.88 & 600m \\ 
アオキ  &  Aj &  21 & 76.20 & 2.81 & 0.82 & 常緑樹 & 4.57 & 低木 & 0.85 & 600m \\ 
アオハダ  &  Im &  23 & 73.90 & 3.35 & 0.84 & 落葉樹 & 10.84 & 亜高木 & 0.87 & 800m \\ 
ヒサカキ  &  Ej &  22 & 68.20 & 3.09 & 0.93 & 常緑樹 & 6.72 & 亜高木 & 0.93 & 600m \\ 
ウラジロガシ  &  Qs &  21 & 66.70 & 1.81 & 0.98 & 常緑樹 & 16.00 & 高木 & 0.95 & 600m \\ 
シキミ  &  Ia &  30 & 63.30 & 3.40 & 0.97 & 常緑樹 & 8.81 & 亜高木 & 0.96 & 800m \\ 
イヌツゲ  &  Ic &  26 & 53.80 & 1.38 & 0.99 & 常緑樹 & 6.79 & 亜高木 & 0.95 & 800m \\ 
マメザクラ  &  Ci &  23 & 47.80 & 1.57 & 0.90 & 落葉樹 & 9.89 & 亜高木 & 0.86 & 700m \\ 
アカガシ  &  Qa &  27 & 37.00 & 1.44 & 0.91 & 常緑樹 & 17.89 & 高木 & 0.89 & 600m \\ 
カジカエデ  &  Ad &  25 & 28.00 & 0.84 & 0.97 & 落葉樹 & 12.99 & 亜高木 & 0.90 & 800m \\ 
シラキ  &  Ns &  25 & 24.00 & 0.60 & 1.00 & 常緑樹 & 7.44 & 亜高木 & 0.98 & 600m \\ 
タンナサワフタギ  &  Sc &  23 & 13.00 & 0.13 & 0.99 & 落葉樹 & 6.83 & 亜高木 & 0.99 & 800m \\ 
\rowcolor{Blue2!30}イロハモミジ  &  Ap &  21 & 4.80 & 0.05 & 0.98 & 落葉樹 & 16.41 & 高木 & 0.94 & 600m \\ 
\rowcolor{Blue2!30}イタヤカエデ  &  Am &  24 & 4.20 & 0.04 & 0.99 & 落葉樹 & 16.91 & 高木 & 1.00 & 700m \\ 
\rowcolor{Blue2!30}シラキ  &  Nj &  27 & 3.70 & 0.04 & 0.98 & 落葉樹 & 7.48 & 亜高木 & 1.00 & 800m \\ 
\rowcolor{Blue2!30}ブナ  &  Fc &  27 & 0.00 & 0.00 & 1.00 & 落葉樹 & 23.12 & 高木 & 1.00 & 700m \\ 
\rowcolor{Blue2!30}ケヤキ  &  Zs &  27 & 0.00 & 0.00 & 1.00 & 落葉樹 & 20.52 & 高木 & 1.00 & 700m \\ 
\rowcolor{Blue2!30}オオモミジ  &  Aa &  26 & 0.00 & 0.00 & 1.00 & 落葉樹 & 16.27 & 高木 & 1.00 & 700m \\ 
\rowcolor{Blue2!30}ヒメシャラ  &  Sm &  21 & 0.00 & 0.00 & 1.00 & 落葉樹 & 17.97 & 高木 & 1.00 & 800m \\ 
\rowcolor{Blue2!30}イヌシデ  &  Ct &  28 & 0.00 & 0.00 & 1.00 & 落葉樹 & 17.64 & 高木 & 1.00 & 800m \\ 
  \hline\end{tabular}}
\end{wrapfigure}

\textbf{\underline{複幹率}}... 平均$\pm$標準偏差\\
{\footnotesize 亜高木のチドリノキ、アブラチャンでは100\%、ブナやケヤキなど、すべての高木落葉樹種は0\%}\\
\textbf{\underline{平均幹数}}... 平均$\pm$標準偏差\\
{\footnotesize 最大はチドリノキの8.27本}\\
\textbf{\underline{AUC}}... 平均$\pm$標準偏差\\
\textbf{\underline{主幹と全萌芽幹のBA比}}... 平均$\pm$標準偏差

\vspace{1em}
\renewcommand{\baselinestretch}{0.8}
\begin{mdframed}[style=subsection.frame]
% 幹の被陰状況
  \huge\textbf{\color{Black1}{\Large{\ding{194}萌芽幹への資源分配率が高い種ほど株内での自己被陰率が下がる}}}
\end{mdframed}
\renewcommand{\baselinestretch}{1.2}

多くの萌芽幹は主幹を中心とした個体内の他の幹によって被陰される
% Shade conditionとAUCだけの図にする
% 萌芽幹の被陰状況を示す図(必要?)

\begin{figure}
	\begin{center}
		\includegraphics[scale=0.8]{/Users/uri/Dropbox/LAB/Research/2013Structure_of_Multi-stemmed_Trees/Slide/talk_150206WS/images/Figure3.pdf}
		\includegraphics[scale=0.6]{/Users/uri/Dropbox/LAB/Research/2013Structure_of_Multi-stemmed_Trees/Slide/talk_150206WS/images/Figure4.pdf}
	\end{center}
\end{figure}

\columnbreak
\renewcommand{\baselinestretch}{0.8}
\begin{mdframed}[style=subsection.frame]
% 樹高と萌芽特性
  \huge\textbf{\color{Black1}{\Large{\ding{193}平均樹高と萌芽特性は強い相関を示す}}}
\end{mdframed}
\renewcommand{\baselinestretch}{1.2} %revert baseline

樹高に対する萌芽特性との関係(Spearmanの順位相関係数)

\textbf{\underline{複幹率}}... \\
\textbf{\underline{平均幹数}}... \\
\textbf{\underline{AUC}}... \\
\textbf{\underline{主幹と全萌芽幹のBA比}}...

\begin{figure}
	\begin{center}
		\includegraphics[scale=1.0]{/Users/uri/Dropbox/LAB/Research/2013Structure_of_Multi-stemmed_Trees/Slide/talk_150206WS/images/Figure1.pdf}
	\end{center}
\end{figure}

系統独立対比による解析でも同様の傾向を検出

\begin{figure}
	\begin{center}
	  \includegraphics[scale=1.0]{/Users/uri/Dropbox/LAB/Research/2013Structure_of_Multi-stemmed_Trees/Slide/talk_150206WS/images/Figure2.pdf}
	\end{center}
\end{figure}

\end{multicols}
%%%%%%%%%%%%%%%%%%%%%%%%%%%%%%%%%%%%%%%%%%%%%%%%%%%%%%%%%%%%
\begin{mdframed}[style=section.frame]
  \centering\huge\textbf{\color{white}{\faWrench \vspace{0.02em} 方法}}
\end{mdframed}

\begin{multicols}{3}\footnotesize{

\subsection*{\small{調査地: 静岡県函南原生林}}

\begin{itemize}\setlength{\itemindent}{1em}
  \item 禁伐指定により人為的撹乱の影響が少ない成熟林
  \item 3つの標高地点にプロット(1.0ha)を設置
  \begin{itemize}\setlength{\itemindent}{1em}
    \item 常緑樹が優占する600m, 落葉樹優占の800m, \newline 常緑・落葉移行帯の700m
    \item 胸高直径(DBH) 5cm以上を対象に毎木調査
  \end{itemize}
\end{itemize}

%\begin{figure}
%  \includegraphics[scale=0.09]{/Users/uri/Dropbox/LAB/Research/2012QuercusLHS/Images/Studysite_and_area.pdf}
%\end{figure}

\subsection*{\small{対象種および対象個体}}

\begin{itemize}\setlength{\itemindent}{1em}
  \item プロット内で個体数の多い25種
  \begin{itemize}\setlength{\itemindent}{1em}
    \item 落葉樹15種、常緑樹10種
    \item 高木、亜高木、低木
  \end{itemize}
  \item 各種のDBH上位60 - 90\%の成熟個体
\end{itemize}

\columnbreak
\subsection*{\small{測定項目}}

\begin{figure}
  \center\includegraphics[scale=0.82]{/Users/uri/Dropbox/LAB/Research/2012QuercusLHS/Images/definition_of_basal_sprout.pdf}
\end{figure}


AUC... 

\columnbreak
\subsection*{\small{系統解析}}

%\vspace{-3em}
\begin{wrapfigure}{r}{16em}
  \includegraphics[scale=0.8]{/Users/uri/Dropbox/git/Data-analyses/life-history_traits_and_stool_structure/Figure/kn25_phylogenetic_tree_phylogram-1.pdf}
\end{wrapfigure}

系統樹を作成、

解析時、系統独立対比を実施

}\end{multicols}


% https://github.com/brunobeltran/MatrixExp
% https://github.com/thmosqueiro/sciposterpp
% http://latexbr.blogspot.jp/2011/07/posters-cientificos-no-latex.html
% http://hikaru.6.ql.bz/memo.php?id=latex#sciposter
\end{document}