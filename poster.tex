\documentclass[a0, 30pt, plainboxedsections]{sciposter} % Available size are 14, 17, 20, 25, 30, 36...
\usepackage{zxjatype} \setjamainfont{YuGo-Medium} %YuGo-Medium
\usepackage{fontspec, fontawesome}
\usepackage[utf8]{inputenc}
%\usepackage{geometry} \geometry{top=1cm, left=2cm, right=6cm}
\usepackage{amsmath, amssymb}
\usepackage{multicol}
\usepackage{tikz}
\usepackage[framemethod=tikz]{mdframed}
\usepackage{color, xcolor}
%\usepackage[american]{babel} %american, english
%\usepackage[ansinew]{inputenc}
%\usepackage[Mandalore]{sciposterpp}
\usepackage{setspace}
\usepackage{wasysym}
\usepackage[scaled=0.92]{helvet}
\usepackage{pifont}
\usepackage{colortbl}
\usepackage{float, wrapfig}
\usepackage[format=plain,labelformat=simple,labelsep=period,font=scriptsize]{caption}
\usepackage[T1]{fontenc}

%\renewcommand{\rmdefault}{qtm} % serif font
%\renewcommand{\titlesize}{\Huge}
\renewcommand{\authorsize}{\Large} % title, author, inst
\renewcommand{\baselinestretch}{1.2}
%%%%%%%%%%%%%%%%%%%%%%%%%%%%%%%%%%%%%%%%%%%%%%%%%%%%%%%%%%%%
% define color
\definecolor{mainCol}{RGB}{255, 255, 250} % BACKGROUND
\definecolor{TextCol}{HTML}{469B9A}
\definecolor{Black1}{HTML}{0D1412}
\definecolor{Blue1}{HTML}{469B9A} \definecolor{Blue2}{HTML}{84C0CC}
\definecolor{Orange1}{HTML}{F9B04F}
\definecolor{Grey1}{HTML}{D1D1D3}
%%%%%%%%%%%%%%%%%%%%%%%%%%%%%%%%%%%%%%%%%%%%%%%%%%%%%%%%%%%%
% define font awesome
\providecommand\faUniv{{\FA\symbol{"F19C}}}
% define mdframed style
\mdfdefinestyle{section.frame}{outerlinewidth=0, roundcorner=8pt, backgroundcolor=Blue1, linecolor=Blue1}
\mdfdefinestyle{subsection.frame}{outerlinewidth=6, roundcorner=8pt, backgroundcolor=Blue1!20, linecolor=Blue2}
\mdfdefinestyle{conclusion.frame}{outerlinewidth=8, roundcorner=8pt, backgroundcolor=Orange1!30,  linecolor=Orange1}
\mdfapptodefinestyle{subsection.frame}{rightline=true, innerleftmargin=16, innerrightmargin=16, 
frametitlerule=true, frametitlebackgroundcolor=Blue2}
\mdfapptodefinestyle{conclusion.frame}{rightline=true, innerleftmargin=16, innerrightmargin=16, 
frametitlerule=true, frametitlebackgroundcolor=Orange1}
% define multicol style
\setlength{\linewidth}{18pt}
\setlength{\columnsep}{2cm}
\def\columnseprulecolor{\color{Grey1}}
% define ruby
% ref) http://yasuda.homeip.net/insomnia/2011/07/sitatuki-ruby.html
\newcommand\uruby[3][0]{\leavevmode
  % 親文字とルビの寸法を取得
  \setbox0=\hbox{#2}\setbox1=\hbox{\tiny #3}%
  % 幅の大きいほうの寸法を \dimen0 に格納
  \ifdim\wd0>\wd1 \dimen0=\wd0\else\dimen0=\wd1\fi
  % \dimen1 に「ルビ高さ+深さ+間隔値」(下にずらす量)を設定
  \dimen1=\ht1 \advance\dimen1 \dp1 \advance\dimen1 #1\relax
  % \dimen0 の幅で親文字を出力し,ルビを \dimen1 寸法だけ下に下げる
  \hbox to\dimen0{\hfil#2\hfil}%
  \kern-\dimen0\raise-\dimen1\hbox{\vbox{\hbox to\dimen0{\tiny #3}}}}%

\renewcommand{\figurename}{図}
\renewcommand{\tablename}{表}

%%%%%%%%%%%%%%%%%%%%%%%%%%%%%%%%%%%%%%%%%%%%%%%%%%%%%%%%%%%%
\rightlogo{images/logo.pdf}
\title{\textcolor{Blue1}{函南原生林を構成する樹種の株構造:\\\vspace{-0.2em} 葉群配置戦略としての多様な萌芽性}}
\author{\underline{瓜生真也}, 鄭 欣怡, 酒井暁子 (横浜国立大・院・環境情報) \normalsize{\faEnvelope \hspace{0.02em} \fontspec{GillSans-Italic}{suika1127@gmail.com}}}
%\institute{} \email{}
%\nologos
%%%%%%%%%%%%%%%%%%%%%%%%%%%%%%%%%%%%%%%%%%%%%%%%%%%%%%%%%%%%
\begin{document}
\conference{150319 日本生態学会第62回全国大会@鹿児島 [PA1-042]}
\maketitle
%%%%%%%%%%%%%%%%%%%%%%%%%%%%%%%%%%%%%%%%%%%%%%%%%%%%%%%%%%%%
\vspace{-2em}
\begin{mdframed}[style=section.frame]
  \centering\LARGE\textbf{\color{white}{はじめに}}
\end{mdframed}

\begin{multicols}{2}

\subsection*{目的: 安定した成熟林を構成する樹種を対象に、\\株構造と他の性質との関係を明らかにする}

樹木のサイズを決める樹高と幹の太さは、光獲得競争と物理的安定性や水輸送効率によって決まり、形質間のトレードオフがある。

森林内では主に高木種が単幹となる一方、低木種を中心に複数の幹をもつ種や個体が共存しており、こうした\textbf{\underline{株構造の違いは種内間でみられる}}(例えばEscandón \textit{et al.} 2013; Bay \textit{et al.} 2014)。

\columnbreak
\subsection*{仮説: 複数の幹をもつ種は...}

\begin{enumerate}\setlength{\itemindent}{1em}
\item 単幹で生育する種よりも\underline{\textbf{樹高が低く}}なる
\item 多幹化に伴う\underline{\textbf{自己被陰の影響を軽減できる葉群配置}}を行う
\end{enumerate}

\begin{figure}
  \centering\includegraphics[scale=0.8]{images/sprout_merit_demerit.pdf}
\end{figure}
\end{multicols}
%%%%%%%%%%%%%%%%%%%%%%%%%%%%%%%%%%%%%%%%%%%%%%%%%%%%%%%%%%%%
\begin{mdframed}[style=section.frame]
  \centering\LARGE\textbf{\color{white}{方法}}
\end{mdframed}

\begin{multicols}{4}\footnotesize{

\subsection*{\small{調査地: 静岡県函南原生林}}

\begin{itemize}\setlength{\itemindent}{1em}
  \item 江戸時代からの禁伐指定により、人為的撹乱の影響が少ない成熟林
  \item 3つの標高地点にプロット(1.0ha)を設置
  \begin{itemize}\setlength{\itemindent}{1em}
    \item 600m(常緑樹優占)、700m(移行帯)、800m(落葉樹優占)
    \item 胸高直径(DBH) 5cm以上を対象に毎木調査\footnote{プロットの設置、毎木調査は東京農業大学 武生、澤田(現在、鹿児島大学)、国士舘大学 磯谷、横浜国立大学 吉田らの調査による}
  \end{itemize}
\end{itemize}

\subsection*{\small{対象種および対象個体の選別}}

\begin{itemize}\setlength{\itemindent}{1em}
  \item 1プロット内で20個体以上出現した種
  \begin{itemize}\setlength{\itemindent}{1em}
    \item 14科17属25種(常緑樹10種、落葉樹15種)
  \end{itemize}
  \item 各種のDBH上位60 - 90\%の成熟個体
  \item 各種について個体数が最も多い標高でサンプリング(cf. 結果 \textbf{表1})
\end{itemize}

\columnbreak
\subsection*{\small{測定項目: 幹のサイズ、萌芽幹の被陰状態}}

\begin{figure}
	\centering
	\includegraphics[scale=0.7]{images/stem_type_define.pdf}
\end{figure}

各萌芽幹については、被陰状況を次の\ding{192}から\ding{194}の\\いずれかから記録

\begin{figure}
	\centering
	\includegraphics[scale=0.7]{images/shade_category.pdf}
\end{figure}

\columnbreak
\subsection*{\small{株構造の定量化: \uruby{株構造指数}{Stool Structure Index(SSI)}}}

\flushright{\tiny Uryu \textit{et al.} (Unpublished)}

幹本数とサイズの両方を反映。0.5から1の\\値をとる。\flushleft

\begin{figure}
  \centering
   \includegraphics[scale=0.9]{images/tmp1.pdf}
\end{figure}

個体(株)\footnote{(株)は株式会社の略ではありません}の中でBAが大きい順に幹を並べる。なお最も大きい幹は主幹となる。

\begin{itemize}
  \item 0.5: 主幹を含めてすべての幹が同じサイズ
  \item 1: 主幹のみからなる個体
\end{itemize}

\columnbreak
\subsection*{\small{系統独立対比による種間の比較}}

APG I\hspace{-1pt}I\hspace{-1pt}Iのデータベース\footnote{R20120829.new (\url{https://github.com/camwebb/tree-of-trees/blob/master/megatrees/R20120829.new)}}をもとにPhylomatic\footnote{Webb and Donoghue 2005(\url{http://phylodiversity.net/phylomatic/})}にて系統データを作成。

\begin{figure}
 \begin{minipage}{0.5\hsize}
  \centering
   %\includegraphics[scale=0.5]{/Users/uri/Dropbox/git/Data-analyses/life-history_traits_and_stool_structure/Figure/kn25_phylogenetic_tree_cladogram-1.pdf}
 \end{minipage}
 \begin{minipage}{0.5\hsize}
  \centering
   \includegraphics[scale=0.5]{/Users/uri/Dropbox/git/Data-analyses/life-history_traits_and_stool_structure/Figure/kn25_phylogenetic_tree_cladogram-1.pdf}   
 \end{minipage}
\end{figure}


}\end{multicols}
%%%%%%%%%%%%%%%%%%%%%%%%%%%%%%%%%%%%%%%%%%%%%%%%%%%%%%%%%%%%
\begin{mdframed}[style=section.frame]
  \centering\LARGE\textbf{\color{white}{結果}}
\end{mdframed}

\begin{multicols}{2}

\renewcommand{\baselinestretch}{0.8}
\begin{mdframed}[style=subsection.frame,frametitle={\textbf{\LARGE{\ding{192}}\Large{25種の株構造は種間で大きく異なる}}}]
\renewcommand{\baselinestretch}{1.2}

\subsection*{各種の株構造と全種の萌芽特性の平均および標準偏差}

\begin{wraptable}{r}{15.3em}\caption{複幹率が高い種種の順で並べた調査対象種の要約}\vspace{-0.4em}
  \scalebox{0.44}{\begin{tabular}{llrrrrl}
  \hline
\rowcolor{white}種名 & 個体数 & 複幹率 & 萌芽本数 & 株構造指数 & 萌芽幹RBA & 標高 \\ 
  \hline
\rowcolor{Orange1!30}チドリノキ &  22 & 100.00 & 8.27 & 0.86 & 0.43 & 700 \\ 
\rowcolor{Orange1!30}アブラチャン &  31 & 100.00 & 5.87 & 0.69 & 0.62 & 800 \\ 
\rowcolor{Blue2!80}ヒイラギ &  22 & 81.80 & 2.50 & 0.91 & 0.09 & 600 \\ 
\rowcolor{Blue2!80}イヌガシ &  27 & 81.50 & 4.52 & 0.96 & 0.18 & 600 \\ 
\rowcolor{Orange1!30}カマツカ &  20 & 80.00 & 1.55 & 0.49 & 0.21 & 800 \\ 
\rowcolor{Blue2!80}ヤブニッケイ &  27 & 77.80 & 2.15 & 0.88 & 0.12 & 600 \\ 
\rowcolor{Blue2!80}アオキ &  21 & 76.20 & 2.81 & 0.85 & 0.18 & 600 \\ 
\rowcolor{Orange1!30}アオハダ &  23 & 73.90 & 3.35 & 0.87 & 0.16 & 800 \\ 
\rowcolor{Blue2!80}ヒサカキ &  22 & 68.20 & 3.09 & 0.93 & 0.07 & 600 \\ 
\rowcolor{Blue2!80}ウラジロガシ &  21 & 66.70 & 1.81 & 0.95 & 0.02  & 600 \\ 
\rowcolor{Blue2!80}シキミ&  30 & 63.30 & 3.40 & 0.96 & 0.03 & 800 \\ 
\rowcolor{Blue2!80}イヌツゲ &  26 & 53.80 & 1.38 & 0.95 & 0.01  & 800 \\ 
\rowcolor{Orange1!30}マメザクラ &  23 & 47.80 & 1.57 & 0.86 & 0.10  & 700 \\ 
\rowcolor{Blue2!80}アカガシ &  27 & 37.00 & 1.44 & 0.89 & 0.09  & 600 \\ 
\rowcolor{Orange1!30}カジカエデ &  25 & 28.00 & 0.84 & 0.90 & 0.03  & 800 \\ 
\rowcolor{Blue2!80}シロダモ &  25 & 24.00 & 0.60 & 0.98 & 0.00  & 600 \\ 
\rowcolor{Orange1!30}タンナサワフタギ &  23 & 13.00 & 0.13 & 0.99 & 0.01  & 800 \\ 
\rowcolor{Orange1!30}イロハモミジ &  21 & 4.80 & 0.05 & 0.94 & 0.02  & 600 \\ 
\rowcolor{Orange1!30}イタヤカエデ &  24 & 4.20 & 0.04 & 1.00 & 0.01  & 700 \\ 
\rowcolor{Orange1!30}シラキ &  27 & 3.70 & 0.04 & 1.00 & 0.02 & 800 \\ 
\rowcolor{Orange1!30}ブナ &  27 & 0.00 & 0.00 & 1.00 & 0.00 & 700 \\ 
\rowcolor{Orange1!30}ケヤキ &  27 & 0.00 & 0.00 & 1.00 & 0.00 & 700 \\ 
\rowcolor{Orange1!30}オオモミジ &  26 & 0.00 & 0.00 & 1.00 & 0.00 & 700 \\ 
\rowcolor{Orange1!30}ヒメシャラ &  21 & 0.00 & 0.00 & 1.00 & 0.00 & 800 \\ 
\rowcolor{Orange1!30}イヌシデ &  28 & 0.00 & 0.00 & 1.00 & 0.00  & 800 \\ 
   \hline
\end{tabular}}
\end{wraptable}

\textbf{\underline{複幹率}}{\footnotesize: 複幹を持っている個体の割合}\\平均43.43$\pm$標準偏差36.02\\
{\scriptsize 0\%(ブナやケヤキなどの落葉広葉種)から100\%(アブラチャン、チドリノキ)と高い多様性}\\
\textbf{\underline{萌芽本数}}\\2.27$\pm$2.11{\footnotesize(複幹をもつ個体のみで計算)}\\
{\scriptsize 最大はチドリノキの8.27本}\\
\textbf{\underline{株構造指数}}\\0.89$\pm$0.12{\footnotesize(複幹をもつ個体のみで計算)}\\
\textbf{\underline{萌芽幹BA比}}\\0.12$\pm$0.16{\footnotesize(複幹をもつ個体のみで計算)}\\
{\scriptsize 最大はアブラチャンの0.62。0.1未満の種が12種}

\vspace{2em}

\end{mdframed}

\renewcommand{\baselinestretch}{0.8}
\begin{mdframed}[style=subsection.frame,frametitle=\textbf{\LARGE{\ding{194}}\Large{萌芽性が強い種ほど自己被陰率が下がる}}]
\renewcommand{\baselinestretch}{1.2}

\subsection*{複幹率5\%以上の17種を対象とした幹の被陰状況と萌芽性}

多くの萌芽幹は個体内の他の幹により被陰されていた。一方で萌芽性が高い種ほど自己被陰の割合が減少する傾向が系統関係に依存せずみられた。

\begin{multicols}{2}

\begin{figure}
  \centering
   \includegraphics[scale=1.0]{images/tmp2} % size w1855:h1000
   
   {\tiny 図3.萌芽幹の被陰状況割合}
\end{figure}

\begin{figure}
  \centering
   \includegraphics[scale=1.0]{images/fig4self_shad_ssi_cor.pdf}
   
   {\tiny 図4.各種の株構造指数と萌芽幹の自己被陰率との関係。形質値、対比値ともに有意な正の相関関係がみられた。
   
\end{figure}

\end{multicols}

\renewcommand{\baselinestretch}{0.8}
\begin{figure}
 \begin{minipage}{0.5\hsize}
  \centering
   \includegraphics[scale=1.0]{images/tmp2} % size w1855:h1000

   
   \newline
   
 \end{minipage}
 
\end{figure}
\renewcommand{\baselinestretch}{1.2}


\end{mdframed}

\columnbreak
\renewcommand{\baselinestretch}{0.8}
\begin{mdframed}[style=subsection.frame,frametitle=\textbf{\LARGE{\ding{193}}\Large{樹高が高くなるほど萌芽性が強くなる}}]
\renewcommand{\baselinestretch}{1.2} %revert baseline

\subsection*{萌芽特性と樹高との相関関係}

\begin{figure}
	\centering
		\includegraphics[scale=0.8]{images/fig1height_spr_cor.pdf} % w4000;
		
	{\tiny 図1.萌芽に関する4つの指標と樹高との関係。すべて有意な相関(Spearmanの順位相関 \textit{P} < 0.05... *, \textit{P} < 0.01... **, \textit{P} < 0.01... ***)。
   
   \textbf{\underline{凡例}}... 色は\textcolor{Orange1}{落葉樹}:、\textcolor{Blue1}{常緑樹}を示す。\ding{115}: 低木種、\ding{110}: 亜高木種、\ding{108}: 高木種}
\end{figure}

\begin{figure}
	\centering\hspace{-2em}
	  \includegraphics[scale=0.8]{images/fig2height_spr_cor_pic.pdf}\\
	  
	  \hspace{-2em}{\tiny 図2.系統独立対比による萌芽特性と樹高との関係。萌芽幹RBAを除いて相関は有意}
\end{figure}

\end{mdframed}

\renewcommand{\baselinestretch}{1.2}
\begin{mdframed}[style=conclusion.frame,frametitle={\textbf{\large{\faFlagAlt \vspace{0.02em} 結論: {萌芽性は最大樹高と相反して発達しており、群集の中に多様な萌芽性をもつ種が存在する}}}}]
  \vspace{0.4em}
  \flushleft
  \normalsize{\underline{結果\ding{192}}多くの種は複数の幹をもつ\\
  \underline{結果\ding{193}}樹高と萌芽特性はトレードオフ\\
  }
  \large{\faHandLeft \vspace{0.02em} この関係は対象とした種の\textbf{\underline{系統的偏りによって生じた偽相関ではない}}}
  
  \vspace{0.4em}
  \normalsize{\underline{結果\ding{192}\ding{193}}高木種(特に落葉樹)は萌芽幹をもたない、あるいは主幹に対してわずかな大きさしかない\\
  \underline{結果\ding{194}}より萌芽性が強い種では株内の幹による自己被陰を軽減する傾向\\
  }
  \large{\faHandLeft \vspace{0.02em}  群集内で「\textbf{\underline{より上層で光を獲得する}}」と「\textbf{\underline{下層でより効率的に光を}} \textbf{\underline{受ける}}」の戦略分化があり
  、萌芽性は後者の戦術として進化している。}
 
\end{mdframed}

\end{multicols}

% https://github.com/brunobeltran/MatrixExp
% https://github.com/thmosqueiro/sciposterpp
% http://latexbr.blogspot.jp/2011/07/posters-cientificos-no-latex.html
% http://hikaru.6.ql.bz/memo.php?id=latex#sciposter
\end{document}